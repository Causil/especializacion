\section{Datos originales}
Se dispone de dos fuentes de datos.
Fuente de datos 1: se cuenta con la informaci\'on para 1404 d\'ias 
del consumo energ\'ia activa y reactiva para el periodo comprendido 
entre febrero de 2020 y mayo de 2024. Estos datos son proporcionados 
directamente por la empresa prestadora de servicios de energ\'ia el\'ectrica y
son un instrumento para el control del pago de la facturaci\'on mensual. 
Se cuenta con un archivo en excel por a\~no. Cada archivo contiene 
cinco variables de las cuales se tiene informaci\'on para cada hora, 
as\'i:

\begin{enumerate}
    \item Energ\'ia Activa Consumo (kWh).
    \item Energ\'ia Activa Generaci\'on (kWh).
    \item Energ\'ia Reactiva Inductiva (kVarh).
    \item Energ\'ia Reactiva Capacitiva (kVarh).
    \item Fecha.
\end{enumerate}


Fuente de datos 2: Mediciones de los sensores de equipos de la empresa productora de 
l\'acteos. Los datos de los sensores est\'an alojados en DynamoDB con una actualizaci\'on 
segundo a segundo. Los sensores toman los datos de ciertas m\'etricas asociadas 
al estado de los equipos para el seguimiento en la producci\'on, los cuales son almacenados 
en la nube a trav\'es del servicio de AWS. Actualmente el acceso a las bases de datos se 
encuentra limitado por parte de la oficina de TI. Una vez se restablezca el servicio, se 
proceder\'a a realizar el an\'alisis correspondiente de los datos disponibles. Se est\'a a 
la espera de la entrega del diccionario de datos generados por los sensores por parte del 
equipo de TI. Las variables que contempla esta fuente de datos se relacionan en la tabla 
1 y su an\'alisis se realizar\'a para una entrega posterior.