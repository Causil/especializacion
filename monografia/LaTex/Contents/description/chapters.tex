\setlength{\parindent}{15pt} Una empresa de productos l\'acteos requiere disminuir los costos asociados 
al consumo de energ\'ia en una planta de producci\'on. Se aplicar\'an modelos
para pronosticar la generaci\'on de energ\'ia en una planta de producci\'on. Se aplicar\'an 
modelos para pronosticar la generaci\'on de energ\'ia reactiva que afecta el consumo pero no agrega valor
a la empresa. Se cuenta con datos historial de consumo de energ\'ia el\'ectrica desde el a\~no 2020 y con datos
capturados por sensores instalados en las m\'aquinas de la planta desde el a\~no 2023. Las m\'etricas de desempe\~no son las 
asociadas a la precisi\'on de los modelos y el valor de la facturaci\'on mensual.

\section{Problema de negocio}
\setlength{\parindent}{15pt} Una empresa de productos l\'acteos requiere disminuir los costos 
asociados al consumo de energ\'ia en una planta de producci\'on.

Seg\'un \textbf{CELSIA} (2022), la energ\'ia reactiva es un tipo de energ\'ia el\'ectrica absorbida o inyectada a la red por algunos equipos que para su funcionamiento necesitan un campo 
magn\'etico, tales como motores, transformadores, ascensores, sistemas de bombeo de agua, motores de aireaci\'on de piscinas, iluminaci\'on eficiente, entro otros. La unidad de medida 
de este ripo de energ\'ia es $kVarh$. En complemento, de acuerdo con (EPM, 2023), la energ\'ia reactiva se puede entender como una energ\'ia que ocupa espacio de las redes el\'ectricas, pero 
no es \'util a la hora de hacer trabajo. Como esta energ\'ia que ocupa espacio de las redes el\'ectricas, pero no es \'util a la hora de hacer trabajo. Como la energ\'ia reactiva 
satura las redes, es necesario para las empresas reducirla a su m\'inima expresi\'on para evitar problemas en la calidad de la energ\'ia, sobrecargas e ineficiencias que redundari\'ia en mayores 
costos para rpestar el servicio.

De acuerdo con lo indicado por el personal t\'ecnico de la Planta, en los \'ultimos a\~nos, el nivel de generaci\'on de energ\'ia reactiva en la planta ha ido en aumento, y aunque la informaci\'on 
de energ\'ia global se conoce, no es posible identificar la totalidad de las fuentes que la est\'an generando, lo cual tiene un impacto significativo sobre los costos de producci\'on.
La energ\'ia reactiva puede ser generada por m\'ultiples fuentes (motores, equipos el\'ectricos, instalaciones, etc.) y puede variar seg\'un el estado de la maquinaria, la tecnolog\'ia de las ma\'aquinas, 
el mantenimiento y otras variables no identificadas, por lo cual no es posible discriminar el aporte de cada fuente al sobre costo establecido por el prestador de servicios de energ\'ia, lo cual dificulta la toma de decisiones 
para optimizar su uso. 