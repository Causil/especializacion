% style sheet for the thesis
% Chapter and section title format
%\titleformat{\section}[block]{\normalfont\huge\bfseries}{\thesection.}{.5em}{\Huge}[{}]
% \titlespacing*{\chapter}{0pt}{-19pt}{25pt}
% \titleformat{\section}[block]{\normalfont\Large\bfseries}{\thesection.}{.5em}{\Large}



% code formatting with listing
%\lstset{
%  basicstyle=\ttfamily,
%  breaklines=true,
%}

% Margins
\geometry{
    a4paper,
    margin=2.75cm
}

% Index depth limit
\setcounter{tocdepth}{2}

% Paragraph Indentation
\setlength{\parindent}{1cm}

%\renewcommand{\lstlistingname}{Code extraction}
%\renewcommand*{\lstlistlistingname}{Code Excerpts Index}

\definecolor{US_red}{cmyk}{0, 1, 0.65, 0.34}
\definecolor{US_yellow}{cmyk}{0, 0.3, 0.94, 0}


%------------configurando el entorno de presentaci\'on de c\'odigo
\lstset{
        tabsize=2, % tab = 2 espacios
        backgroundcolor=\color[HTML]{F0F0F0}, % color de fondo
        captionpos=b, % posición de pie de código, b=debajo
        basicstyle=\ttfamily, % estilo de letra general
        columns=fixed, % columnas alineadas
        extendedchars=true, % ASCII extendido
        breaklines=true, % partir líneas
        prebreak = \raisebox{0ex}[0ex][0ex]{\ensuremath{\hookleftarrow}}, % marcar final de línea con flecha
        showtabs=false, % no marcar tabulación
        showspaces=false, % no marcar espacios
        keywordstyle=\bfseries\color[HTML]{007020}, % estilo de palabras clave
        commentstyle=\itshape\color[HTML]{60A0B0}, % estilo de comentarios
        stringstyle=\color[HTML]{4070A0}, % estilo de strings
}

\lstdefinestyle{abaqusPython}{
        language=python,
		% Palabras clave extra
        morekeywords={CONTINUOUS,NUMBER,MESH,par,name,ParStudy,
		template,define,sample,combine, generate},
		% Delimitadores extra, s porque hay uno a cada lado
        moredelim=[s][\ttfamily\color{magenta}]{<}{>},
}

\lstdefinestyle{abaqusJava}{
        language=python,
		% Palabras clave extra
        morekeywords={CONTINUOUS,NUMBER,MESH,par,name,ParStudy,
		template,define,sample,combine, generate},
		% Delimitadores extra, s porque hay uno a cada lado
        moredelim=[s][\ttfamily\color{magenta}]{<}{>},
}

\lstset{literate=
  {á}{{\'a}}1
  {é}{{\'e}}1
  {í}{{\'i}}1
  {ó}{{\'o}}1
  {ú}{{\'u}}1
  {Á}{{\'A}}1
  {É}{{\'E}}1
  {Í}{{\'I}}1
  {Ó}{{\'O}}1
  {Ú}{{\'U}}1
  {ñ}{{\~n}}1
  {ü}{{\"u}}1
  {Ü}{{\"U}}1
}

\definecolor{bluedistri}{RGB}{59, 160, 242}
\definecolor{orangedistri}{RGB}{250, 112, 56}
%------------configurando el entorno de presentaci\'on de c\'odigo


%\mdfdefinestyle{US_style}{backgroundcolor=US_yellow!20, font=\bfseries, hidealllines=true}

%%%%%%%%%%%%%%%% Para margenes %%%%%%%%%%%
\hfuzz=20pt
\vfuzz=20pt
\hbadness=2000
\vbadness=\maxdimen
%-----------------------------------------%