\chapter{Ingeniero de datos}
Es un profesional que tiene pasi\'on por la automatizaci\'on, confianza en el an\'alisis y los datos. Sus habilidades se enfocan en la 
extracci\'on, transformaci\'on y carga de datos (ETL). Automatizaci\'on de procesos de carga de datos. Procesamiento de datos de diversas fuentes y en c\'omputo
paralelo. Tiene conocimientos en programaci\'on con Python y bases solidas de ingenier\'ia de software, manejo de datos estructurados y no estructurados, c\'omputo, almacenamiento y bases de datos en la nube. 
ETL con herramientas como SQL, Apache Spark, Airflow, Hadoop, AWS, Google Cloud, Azure, entre otros.

\section{M\'odulo 1}

\begin{enumerate}
    \item Ask:
    \item Prepare:
    \item Procesos:
    \item Analyze:
    \item Share:
    \item Act:
\end{enumerate}


\section{Problemas con datos}
Cua\'ndo se preparan los datos para un an\'alisis, se pueden presentar problemas como:
\begin{itemize}
    \item No tenemos los datos que necesitamos.
    \item Los datos que tenemos no son suficientes.
    \item Duplicados
    \item Valores faltantes
    \item Valores at\'ipicos
    \item Datos incorrectos
    \item Datos inconsistentes
    \item Datos desactualizados
    \item Datos no estructurados
    \item Datos no normalizados
    \item Datos no limpios
    \item Datos no estandarizados
    \item Datos no validados
    \item Datos no verificados
    \item Datos no confiables
    \item Datos no seguros
    \item Datos no accesibles
    \item Datos no escalables
    \item Datos no flexibles
    \item Datos no eficientes
    \item Datos no eficaces
\end{itemize}

En muchos casos se suele usar lo que se conoce como datos indirectos en lugar de datos reales. En otros casos, puede no haber un sustituto razonable y la \'unica opci\'on 
ser\'a recopilar m\'as datos.


\subsection{Falta de datos}

posibles soluciones: 

\begin{enumerate}
    \item Recopila datos en menor escala para realizar un an\'alisis preliminar y, luego, pide m\'as tiempo para completar el
     an\'alisis despu\'es de haber recopilado m\'as informaci\'on.
    \item Si no hay tiempo para recopilar informaci\'on, realizar el an\'alisis utilizando datos indirectos de otros conjuntos de datos. Est\'a es la soluci\'on m\'as com\'un.
    
\end{enumerate}


\subsection{Muy pocos datos}

Soluciones posibles:
\begin{enumerate}
    \item Realiza el an\'alisis utilizando datos indirectos junto con datos reales. 
    \item Ajusta tu an\'alisis para alinearlo con los datos que ya tienes.
\end{enumerate}


\subsection{Datos incorrectos, incluidos los datos con errores}

Soluciones posibles:
\begin{enumerate}
    \item Si tienes datos incorrectos porque los requerimientoss no fueron bien interpretados, comunica los requerimientos nuevamente.
    \begin{example}
        \begin{itemize}
            \item Si necesitas datos sobre mujeres y votantes y recibiste datos sobre hombres votantes, vuelve a comunicar qu\'e datos necesitas.
            \item Si los datos se recopilaron incorrectamente, recopila los datos nuevamente.
            \item Si los datos se ingresaron incorrectamente, corrige los datos en la fuente.
            \item Si los datos se procesaron incorrectamente, procesa los datos nuevamente.
            \item Si los datos se analizaron incorrectamente, analiza los datos nuevamente.
            \item Si los datos se presentaron incorrectamente, presenta los datos nuevamente.
        \end{itemize}
    \end{example}
    \item Identifica los errores en los datos y, cuando sea posible, corr\'igelos en la fuente buscando el patr\'on de errores.
        \begin{example}
            \item Si tus datos se encuentran en una hoja de c\'alculo y hay una instrucci\'on condicional o datos booleanos que generan errores en los c\'alculos, modifica la instrucci\'on condicional en lugar de corregir 
            los valores calculados.
        \end{example}
    \item Si no puedes corregir los errores en los datos t\'u mismo, puedes ignorarlos y seguir adelante con el an\'alisis en caso 
          de que el tama\~no de tu muestra a\'un sea lo suficientemente grande como para poder ignorar esos datos y que eso no ocasione un sesgo sistem\'atico.
          \begin{example}
            \item Si tu conjunto de datos es una traducci\'on de otro idioma y alguna traducci\'on no tiene sentido, puedes ignorar los datos con traducciones err\'oneas y seguir adelante con el an\'alisis de los otros datos.
          \end{example}
    \item Nota: a veces los datos con errores pueden ser una se\~nal de advertencia sobre la falta de confiabilidad de los datos. Utiliza tu juicio.
\end{enumerate}

Diagrama de toma de decisiones para recordar c\'omo manejar los errores en los datos o la falta de datos:

\begin{enumerate}
    \item Data Errors:
    % \begin{tikzpicture}[node distance=1.5cm,
    %     every node/.style={fill=white, font=\sffamily}, align=left]
    %   % Specification of nodes (position, etc.)
    %   \node (start)             [activityStarts]              {?`Puedes corregir o pedir un conjunto de datos corregidos?};
    %   \node (onCreateBlock)     [process, below of=start]          {Realiza el an\'alisis despu\'es de que los datos se hayan corregido};
    %   \node (onStartBlock)      [process, below of=onCreateBlock]   {?`Tienes suficientes datos como para poder omitir los datos erroneos?};
    %   \node (onResumeBlock)     [process, below of=onStartBlock]   {Realiza el an\'alisis sin los datos err\'oneos};
    %   \node (activityRuns)      [activityRuns, below of=onResumeBlock] {?`Puedes recopilar m\'as datos?};
    %   \node (onPauseBlock)      [process, below of=activityRuns, yshift=-1cm]
    %                                                                 { Realiza el an\'alisis despu\'es de la recopilaci\'on de datos };
    %   \node (onStopBlock)       [process, below of=onPauseBlock, yshift=-1cm]
    %                                                                  {Modifica el objetivo comercial (de ser posible)};
    %   \node (onDestroyBlock)    [process, below of=onStopBlock, yshift=-1cm] 
    %                                                               {onDestroy()};
    %   \node (onRestartBlock)    [process, right of=onStartBlock, xshift=4cm]
    %                                                               {onRestart()};
    %   \node (ActivityEnds)      [startstop, left of=activityRuns, xshift=-4cm]
    %                                                         {Process is killed};
    %   \node (ActivityDestroyed) [startstop, below of=onDestroyBlock]
    %                                                     {Activity is shut down};
    %   % Specification of lines between nodes specified above
    %   % with aditional nodes for description 
    %   \draw[->]             (start) -- (onCreateBlock);
    %   \draw[->]     (onCreateBlock) -- (onStartBlock);
    %   \draw[->]      (onStartBlock) -- (onResumeBlock);
    %   \draw[->]     (onResumeBlock) -- (activityRuns);
    %   \draw[->]      (activityRuns) -- node[text width=4cm]
    %                                    {Another activity comes in
    %                                     front of the activity} (onPauseBlock);
    %   \draw[->]      (onPauseBlock) -- node {The activity is no longer visible}
    %                                    (onStopBlock);
    %   \draw[->]       (onStopBlock) -- node {The activity is shut down by
    %                                    user or system} (onDestroyBlock);
    %   \draw[->]    (onRestartBlock) -- (onStartBlock);
    %   \draw[->]       (onStopBlock) -| node[yshift=1.25cm, text width=3cm]
    %                                    {The activity comes to the foreground}
    %                                    (onRestartBlock);
    %   \draw[->]    (onDestroyBlock) -- (ActivityDestroyed);
    %   \draw[->]      (onPauseBlock) -| node(priorityXMemory)
    %                                    {higher priority $\rightarrow$ more memory}
    %                                    (ActivityEnds);
    %   \draw           (onStopBlock) -| (priorityXMemory);
    %   \draw[->]     (ActivityEnds)  |- node [yshift=-2cm, text width=3.1cm]
    %                                     {User navigates back to the activity}
    %                                     (onCreateBlock);
    %   \draw[->] (onPauseBlock.east) -- ++(2.6,0) -- ++(0,2) -- ++(0,2) --                
    %      node[xshift=1.2cm,yshift=-1.5cm, text width=2.5cm]
    %      {The activity comes to the foreground}(onResumeBlock.east);
    %   \end{tikzpicture}
    
\end{enumerate}

\section{El tama\~no de la muestra}

La poblaci\'on 

Tama\~no de la muestra(Sample size)
Usar una parte de una poblaci\'on el objetivo es obtener suficiente informaci\'on de un grupo peque\~no de personas para hacer inferencias sobre toda la poblaci\'on.
dentro de una poblaci\'on para formular predicciones o conclusiones sobre la poblaci\'on total. La muestra asegura el grado respecto 
del cual puedes estar confiado en que tus conclusiones representan con presici\'on a la poblaci\'on.

Nota: Cuando utilizas \'unicamente una muestra peque\~na de una poblaci\'on, puede llevar a la incertidumbre en tus conclusiones. No puedes estar el 100\% seguro 
de que tus estad\'isticas son una representaci\'on precisa y completa de la poblaci\'on. Esto lleva a un sesgo del muestreo, ocurre cuando la muestra no es representativa de la 
poblaci\'on en su conjunto. Esto significa que algunos miembros de la poblaci\'on est\'an siendo sobre o subrepresentados. 

Existen m\'etodos para solucionar el sesgo del muestreo.

\begin{enumerate}
    \item Muestreo aleatorio
\end{enumerate}

C\'omo an\'alista de datos es bueno saber que los datos que se van a analizar son representativos de una poblaci\'on y sirven para el objetivo.

