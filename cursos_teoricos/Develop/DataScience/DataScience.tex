

\part{Ciencia de datos}


?`Qu\'e es data ciencia?
Es el proceso de descubrir informaci\'on valiosa de los datos.

?`cu\'al es su finalidad?

\begin{enumerate}
	\item Tomar decisiones y crear estrategias de negocio.
	\item Crear productos de software m\'as inteligentes y funcionales.
\end{enumerate}

?`De que trata este proceso?:
\begin{enumerate}
	\item Obtenci\'on de los datos: a trav\'ez de encuestas
	\item Transformar y limpiar los datos.
	\item Explorar, analizar y visualizar datos.
	\item Usar modelos de machine learning*.
	\item Integrar datos e IA a productos de software.
\end{enumerate}




?`Qu\'e es Data Science?

Data science o ciencia de datos es el proceso de descubrir informaci\'on valiosa de los datos.

?`Cu\'al es su finalidad?
Tomar decisiones y crear estrategias de negocio
Crear productos de software m\'as inteligentes y funcionales.

?`De qu\'e trata este proceso?

Obtenci\'on de los datos
Mediciones
Encuestas
Internet

Transformar y limpiar los datos
Incompletos
Formato Incorrecto

Explorar, analizar y visualizar datos
Patrones o tendencias
Insights
Visualizaciones, gr\'aficos o reportes

Usar modelos de machine learning

Machine learning o aprendizaje autom\'atico es una rama de inteligencia artificial. Su objetivo es que las computadoras aprendan. En machine learning, las computadoras observan grandes cantidades de datos y construyen un modelo capaz de generar predicciones para resolver problemas.

Integrar datos e IA a productos de software
Ponerlos a disposici\'on del usuario final.


La ciencia de datos es una intercepci\'on de conocimiento entre (matem\'aticas y estad\'istica), (ciencias computacionales)  y conocimiento del dominio.

\section{Proyectos data Analysis}
Poner en practica lo mas r\'apido que se pueda, tener proyectos personales, en que gasto los dineros del mes, que productos consigo cada mes, encontrar anomal\'ias, proyectos con kaggle.

\chapter{An\'alisis de datos}

?`Qu\'e es ciencia de datos y big data? ?`C\'omo afectan a mi negocio?

``?`Qu\'e haces en tu trabajo (como cient\'ifico de datos)?
Mi trabajo es crear una soluci\'on matem\'atica o estad\'istica para un problema del negocio''

\section{?`Qu\'e tipo de informaci\'on podemos analizar?}


Descubrir qu\'e tipos de informaci\'on existen, qu\'e industrias los usan y qu\'e tipo de acciones podemos tomar a partir de ellos.

Los principales datos que existen son:

\begin{enumerate}
	\item[Personas:] Este tipo de datos lo extraemos de las personas, es decir lo generamos nosotros cuando le damos like a una foto de facebook, de preferencia, tipo de videos, de quien te gusta mas el contenido, subiendo una foto y etiquetando a un compa\~nero.
	\item[Transacciones:] las monetarias y las no monetarias, cualquier transaci\'on que hago con una tarjeta de cr\'edito o d\'ebito, cuando hacemos un pago electr\'onico o d\'igital  queda una huella, queda un registro de quien lo hizo, por que monto lo hizo y en que establecimiento lo hizo, es muy interesante por que las bancas digitales pueden hacerte recomendaciones sobre el tipo de comercio que te pod\'ia interesar.

		No finacieras: las compa\~nias telef\'onicas identifican cual es tu patr\'on habitual, cuantas llamadas haces, a que personas llamas, cuanto duran tus llamadas, y a partir de esto te llaman para que no abandones el servicio.

	\item[Navegaci\'on web: ] Estas son las famosas cookies, ellas est\'an advirtiendo de la informaci\'on que van a recoger.

	\item[Machine 2 machine: ] Una conexion de una maquina a otra maquina, la usan las plataformas de transporte, google maps y para hacer la locaci\'on entre dispositivos.

	\item[Biom\'etricos: ] Cada vez son mas habituales y \'unicas, huellas digitales, reconocimiento facial.

\end{enumerate}

\section{Flujo de trabajo en ciencia de datos: fases, roles y oportunidades laborales}

Roles en datos:

\begin{description}
	\item[Ingeniero de datos: ] crear bases de datos  Hacer que la empresa, hace la conexion de los dispositivos y las bases de datos,

	\item[Analista business intelligence: ] A partir de la informaci\'on que ha creado el ingeniero de datos va extraer la data, crear cuadros de control, crear dashboard, monitoreo, va automatizar estos procedimientos para que cualquier persona de la empresa pueda interpretarla, las herramientas mas utilizadas son SQL y Excel. No necesariamente sabe Python.

	\item[Data Scientist: ] Sabe hacer el rol del analista, sabe extraer la informaci\'on y sabe predecir, con las herramientas de estad\'istica, nos gu\'ia a donde vamos.

	\item[Data Translator: ] Nos ayuda a proyectar el equipo, nos ayuda a comunicar con los otros equipos del negocio.


\end{description}

\section{Herramientas para cada etapa del an\'alisis de datos}

El primero es el rol del analista y del ingeniero estas son las personas que crean bases de datos y utilizan SQL, se sintetiza la informaci\'on de la base de datos.

El cient\'ifico de datos son herramientas predictivas, son R y Python, R es mas estad\'istico an\'alisis descriptivo,

\section{Python en ciencia de Datos}

Por que numpy para el análisis de datos. Tenemos tres cosas a destacar
\begin{enumerate}
	\item Un poderoso objeto array multidimensional.
	\item Funciones matem
\end{enumerate}

Crear un virtual environments ejecutamos la siguiente linea de comando

\begin{verbatim}
	python3 -m venv my_env
	source bin/activate
\end{verbatim}




\chapter{Proyectos}

\section{Credit Card Fraud Detection}
Anonymized credit card transactions labeled as fraudulent or genuine

	{\LARGE\bf  About Dataset}


	{\large\bf Context}

The dataset contains transactions made by credit cards in September 2013 by European cardholders.
This dataset presents transactions that occurred in two days, where we have 492 frauds out of 284,807 transactions.
The dataset is highly unbalanced, the positive class (frauds) account for 0.172\% of all transactions.

	{\large\bf Content}

It contains only numerical input variables which are the result of a PCA transformation.
Unfortunately, due to confidentiality issues, we cannot provide the original features and more background information about the data.
Features V1, V2, \ldots V28 are the principal components obtained with PCA, the only features which have not been transformed
with PCA are ``Time'' and ``Amount''. Feature ``Time'' contains the seconds elapsed between each transaction and the first transaction
in the dataset. The feature ``Amount'' is the transaction Amount, this feature can be used for example-dependant cost-sensitive learning.
Feature ``Class'' is the response variable and it takes value 1 in case of fraud and 0 otherwise.

Given the class imbalance ratio, we recommend measuring the accuracy using the Area Under the Precision-Recall Curve (AUPRC).
Confusion matrix accuracy is not meaningful for unbalanced classification.

\chapter{Course Structure \& Outline}


Hey everyone chris dutton here and welcome to thinking like an analyst this is a crah course desaigned for

\section{Flavors of Analytics}

All right let's take a minute and talk about the various roles or flavors of analytics because the thing is this field
is very broad and it's very diverse so ir can be helpful to categorize various roles or job titles
based on difertent types of skills so you may have seen venn diagrams out there that look something like this
this version is adapted from learn.co we've got business intelligence skills they're in the blue bubble programming coding skills gray and math  and stats skills in yellow now by
visualizing skills in this way you can map various roles to the diagram based on the overlaps so for instance someone
whi skews towards programmind and math might fall into the machine learning bucket people with and bi skills might fall into data engineering
bi and math and stats maybe we call those people advanced analysts and those who use all three types of skills
relatively equally might fall into the data science category and while this can be helpful to an extent remember that in reality this is fluid
it's flexible and it's often somewhat subjective as well you know you can be a bi analyst who loves stats or programming or a machine
learning engineer with exceptional business intelligence skills or literally any other combination of these categories the key is that this diagram this entire
diagram represents the broader world of data analytics and well many types of roles fall under this analytics umbrella they're
all aligned towards the same ultimate goal using data to make smart decisions. Now think about it that applies whether you're a bi analyst
a statistician a data scientist or an everyday excel jockey and the difference is they come down to things one the types of problems that you're trying to solve
and two the types of tools that you're  using to solve them so in the next lesson we'll dive into one of the most common comparisons out 
there business intelligence versus data  science but for now what's important to specific role you play we're all playing 
for the same team 