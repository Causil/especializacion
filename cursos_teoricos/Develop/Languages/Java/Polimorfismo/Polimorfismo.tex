\chapter{Polimorfismo}

\begin{tikzpicture}
    \begin{class}[text width=6.5cm]{ Funcionario }{0 ,0}
    \attribute{ nombre : String }
    \attribute{ documento : string}
    \attribute{ salario : int}
    \operation{ getBonificacion() }
    \end{class}
\end{tikzpicture}

sqlserver crear tablas el almacen de datos, Donde vas a guardar ese dato? 


\section{This y super}

La palabra clave \href{https://docs.oracle.com/javase/tutorial/java/javaOO/thiskey.html}{\bf this} se refiere al objeto actual en un m\'etodo o constructor. El uso m\'as com\'un de este t\'ermino es eliminar la confusi\'on 
entre los atributos de clase y los par\'ametros del mismo nombre (porque un atributo de clase est\'a sombreado por un pa\'ametro de m\'etodo o constructor).
Adem\'as de este uso, esta palabra se puede utilizar para:

\begin{itemize}
    \item Invoca al constructor de la calse actual;
    \item Invocar el m\'etodo de clase actual;
    \item Devuelve el objeto de calse actual;
    \item Pase un argumento en la llamada al m\'etodo;
    \item Pase un argumento en la llamada al constructor;
\end{itemize}

La palabra clave \href{https://docs.oracle.com/javase/tutorial/java/IandI/super.html}{\bf super} se refiere a objetos de superclase (madre). Se utiliza para llamar a los m'etodos de la superclase y para acceder al constructor de la superclase.
El uso m\'as com\'un de la palabra clase super es eliminar la confuci\'on entre superclases y subclases que tienen m\'etodos con el mismo nombre: 

\begin{description}
    \item[super: ] 
    \begin{itemize}
        \item utilizado para referirse a la variable de instancia de la clase inmediatamente superior (clase madre);
        \item se usa para invocar m\'etodos de la clase inmediatamente superior.
    \end{itemize}
    \item[super(): ] 
    \begin{itemize}
        \item se utiliza para invocar el constructor de la calse inmediatamente supeior (clase madre);
    \end{itemize}
\end{description}

\section{Sobreescritura}

Vimos que la sobreescritura es un concepto importante de la herencia, proque permite redefinir un comportamiento 
previsto en la clase madre a tra\'es de la clase hija. Ahora vea la clase vehiculo abajo.

\begin{lstlisting}
class Vehiculo {
public  void encender() {
    
}
}
\end{lstlisting}

y la clase hija carro: 

\begin{lstlisting}
    class Carro extends Vehiculo {
        // ????
    }
\end{lstlisting}

\section{Modificadores de acceso}

Los modificadores de acceso o accesibilidad son algunas palabras claves utilizadas en el lenguaje Java para definir el nivel de accesiblidad que los elementos de una clase 
(atributos y m\'etodos) e incluso la propia clase puede tener los mismos elementos de otra clase.

\begin{description}
    \item[Public: ] Este es modificador menos restictivo de todos. De esta manera, cualquier componente puede acceder a los miembros de la clase, las clases y las interfaces.
    \item[Protected: ] Al usar este modificador de acceso, los miembros de la case y las clases con accesibles para otros elementos siempre que est\'en dentro del mismo package o, si pertenece a otros packages, siempre que tengan una relaci\'on extendida (herencia), es decir, las clases secundarias peueden acceder a los miembres de su clase principal (o clase de abuelos, etc).
    \item[Private: ] Este es modificador de acceso m\'as restrictivo de todos. Solo se puede acceder a los miebres definidos como privados desde dentro de la clase y desde ning\'un otro lugar, independientemente del paquete o la herencia. 
\end{description}


\href{https://docs.oracle.com/javase/tutorial/java/javaOO/accesscontrol.html}{Referencia en la p\'agina de oracle}

\begin{center}
    \begin{tabular}[t]{l l}
      Interfaz                    &     Clase Abstracta\\ \hline
      Palabra clave: implements   &   	Palabra clase:  extends\\  
      La interfaz admite herencia m\'ultiple 	  &   La clase abstracta no admite herencia m\'ultiple\\  
      Proporciona una abstracci\'on absoluta y no puede tener implementaciones de m\'etodos 	  &  Puede tener m\'etodos con implementaciones        \\   
      La interfaz no contiene constructor 	  & La clase abstracta contiene constructor 	                \\ \hline
      La interfaz no puede tener modificador de acceso, por defecto todos los accesos son p\'ublicos & La clase abstracta puede tener modifiadore sde acceso\\ \hline
      Los miembros de la interfaz pueden ser static   & Solo los miembros completamente abstractos pueden ser static \\hline
    \end{tabular}
  \end{center}