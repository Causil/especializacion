\chapter{Javascript Basic}

\section{JavaScript statements}


A computer program is a list of ``instructions'' to be ``executed'' by a computer.

in a programaming language, these programming are called statements.

A javaScript program is a list of programming statements.

in Html, javaScript programs are excuted by the web browser. 

JavaScript statements are composed of: 

\begin{itemize}
    \item Values
    \item Operators
    \item Expressions
    \item Keywords
    \item Comments
\end{itemize}
This statements tells the browser to write ``Hello Dolly.'' inside an HTML element with id=``demo'';
\begin{verbatim}
<!DOCTYPE html>
<html>
<body>

<h2>JavaScript Statements</h2>

<p>In HTML, JavaScript statements are executed by the browser.</p>

<p id="demo"></p>

<script>
document.getElementById("demo").innerHTML = "Hello Dolly.";
</script>

</body>
</html>
\end{verbatim}
Most JavaScript programs contain many JavaScript statements. The statements are executed, one by one, in the same operadores
as they are written.

JavaScript programs (and javaScript statements) are often called JavaScript code.


Para que JavaScript detecte que las lineas de c\'odigo de nuestro programa son diferentes se utiliza el semicolons ;
al final de la expresi\'on o linea de c\'odigo. Para una mejor legibilidad a los programadores a menudo les gusta evitar las l\'ineas de c\'odigo de m\'as de 80 caracteres.


Si una declaraci\'on de JavaScript no cabe en una l\'inea, el mejor lugar para dividirla es despu\'es de un operador. 


\section{Estructuras del lenguaje}
\subsection{Declaraciones JavaScript}

Las declaraciones en JavaScript est\'an compuestas de:
\begin{enumerate}
    \item Valores.
    \item Operadores.
    \item Expresiones.
    \item Palabras claves.
    \item Comentarios.
\end{enumerate}

Por ejemplo: 
\begin{verbatim}
    let a, b, c;
    a = 5;
    b = 6;
    c = a + b;

function myFunction() {
document.getElementById("demo1").innerHTML = "Hello Dolly!";
document.getElementById("demo2").innerHTML = "How are you?";
}
\end{verbatim}


JavaScript ignora los multiples espacios. 


\subsection{Palabras claves}

Para saber de las palabras claves te invito a visitar el siguiente link \url{https://www.w3schools.com/js/js_reserved.asp}{keywords}.

En general las palabras reservadas son en su mayor\'ia palabras utilizadas por el lenguaje de JavaScript.

\subsection{Valores en JavaScript}

En la sistaxis se definen dos tipos de valores: 

\begin{itemize}
    \item Valores fijos, llamados literales.
    \item Valores variables, son llamados variables.
\end{itemize}

Los n\'umeros decimales, enteros y textos hacen parte de


\subsection{Operadores}
Los diferentes tipos de operadores de JavaScript son: 

\begin{itemize}
    \item Operadores aritm\'eticos.
    \item Operadores de asignaci\'on.
    \item Operadores de comparaci\'on.
    \item Operadores l\'ogicos.
    \item Operadores condicionales.
    \item Operadores de tipo.
\end{itemize}
\subsubsection{Operadores aritm\'eticos}
\begin{flushleft}
\begin{spacing}{1.7}
    \begin{tabular}{p{4cm} ll}
     \textbf{Operador} & \textbf{ Descripci\'on}  \\ % Here the title of your work \\
	  + & Sumar \\
	  - & Restar \\
	  * & Multiplicaci\'on\\
      ** & Exponenciaci\'on \\
      / & Divisi\'on \\
      \% & Modulo\\
      ++ & Incremento\\
      -- & Decremento 
    \end{tabular}
\end{spacing}
\end{flushleft}

Ejemplos: 


\subsubsection{Operadores Asignaci\'on}
\begin{flushleft}
\begin{spacing}{1.7}
    \begin{tabular}{p{4cm} ll}
     \textbf{Operador}      & \textbf{ Ejemplo} & \textbf{ Lo mismo} \\ % Here the title of your work \\
	 $ =$ &       $x = y$       &                     \\
	 $+=$ &       $x += y$      &      $x = x + y$    \\
	 $-=$ &       $x -= y$      &      $ x = x - y$   \\
     $*=$ &       $x *= y$      &      $ x = x * y$   \\
     $ /$ &       $x /= y$      &      $ x = x / y$   \\
     $\%=$&       $x \%= y$     &      $ x = x \% y$  \\
     $**=$ &      $x **= y$     &      $ x = x**y$           
    \end{tabular}
\end{spacing}
\end{flushleft}

\subsubsection{Operadores de comparaci\'on}
\begin{flushleft}
\begin{spacing}{1.7}
    \begin{tabular}{p{4cm} ll}
     \textbf{Operador}      & \textbf{Descripci\'on}  \\ % Here the title of your work \\
        $==$  & 	equal to   \\
        $===$  & 	equal value and equal type   \\
        $!=$  & 	not equal   \\
        $!==$  & 	not equal value or not equal type   \\
        $>$  & 	greater than   \\
        $<$  & 	less than   \\
        $>=$  & 	greater than or equal to   \\
        $<=$  & 	less than or equal to   \\
        $?$  & 	ternary operator   \\
    \end{tabular}
\end{spacing}
\end{flushleft}
Ejemplos: 

\begin{verbatim}
    si consideramos: 
    let x = 8, y=9;
    
    Por consola comparamos podemos obtener de: 
    x == y --> false
    x != y --> true
\end{verbatim}

\subsubsection{Operadores l\'ogicos}
\begin{flushleft}
    \begin{spacing}{1.7}
        \begin{tabular}{p{4cm} ll}
         \textbf{Operador} & \textbf{Descripci\'on}  \\ % Here the title of your work \\
         \&\&              &	logical and   \\
         ||                &	logical or    \\
         !                 &	logical not   \\
        \end{tabular}
    \end{spacing}
\end{flushleft}

\subsubsection{Operadores de tipo}
\begin{flushleft}
    \begin{spacing}{1.7}
        \begin{tabular}{p{4cm} ll}
        \textbf{Operador} & \textbf{Descripci\'on}  \\ % Here the title of your work \\
        typeof            & 	Returns the type of a variable \\
        instanceof        & 	Returns true if an object is an instance of an object type \\
        \end{tabular}
    \end{spacing}
\end{flushleft}


\subsubsection{Bitwise Operators}
Bit operators work on 32 bits numbers.
Any numeric operand in the operation is converted into a 32 bit number. The result is converted back to a JavaScript number.
\begin{flushleft}
    \begin{spacing}{1.7}
        \begin{tabular}{p{4cm} ll}
      %  \textbf{Operador} & \textbf{Descripci\'on} & \textbf{Ejemplo} & \textbf{Lo mismo} & \textbf{Resultado} & \textbf{Decimal}  \\ % Here the title of your work \\
      %  \&                & 	AND                    &  	5 \& 1        &	0101 \& 0001 	&   0001 &	1  \\
      %  \|                 & 	OR                     &  	5 \| 1         &	0101 \| 0001 	&   0101 &	5  \\
      %  \~                 & 	NOT                    &  	\~ 5           &	 \~0101          &	1010 & 10  \\
      %  \^                 & 	XOR                    &  	5 \^ 1         &	0101 \^ 0001     &   0100 &	4  \\
      %  <<                & 	left  shift            &    5 << 1        &	0101 << 1       &	1010 & 10  \\
      %  >>                & 	right shift            &   	5 >> 1        &	0101 >> 1       &	0010 &  2  \\
      %  >>>               &  unsigned right shift      &    5 >>> 1       &	0101 >>> 1      &	0010 &	2  \\
        \end{tabular}
    \end{spacing}
\end{flushleft}

\section{Data types}

JavaScript has 8 Datatypes

\begin{itemize}
    \item String
    \item Number
    \item Bigint
    \item Boolean
    \item Undefined
    \item Null
    \item Symbol
    \item Object
\end{itemize}

The Object Datatype, the object data type can contain: 

\begin{itemize}
    \item An object
    \item An array
    \item A date
\end{itemize}











