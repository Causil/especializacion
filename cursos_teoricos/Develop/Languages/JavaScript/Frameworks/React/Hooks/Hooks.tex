\section{Hooks}
Los hooks son una nueva caracter\'istica en React 16.8. Estos me permiten usar el estado y otras caracteristicas de React sin escribir un clase.

Los hooks resuelven una amplia variedad de problemas aparentemente desconectados en React que hemos encontrado durante m\'as de conco a\~nos de escribir
mantener decenas de miles de componentes.

Esta secc\'on es tomada de \href{https://es.reactjs.org/docs/hooks-intro.html}{\blue Documentaci\'on de React}

En lo siguiente veremos unos tantos hooks.
\input{Develop/Languages/JavaScript/Frameworks/React/Hooks/useState/useState.tex}
\subsection{useContext}
\href{https://dmitripavlutin.com/react-context-and-usecontext/}{useContext}

\subsection{useRef}

useRef es un React hook que te permite hacer referencia a un valor que no es necesario para renderizar.

Le aplica una referencia a los elemetos del DOM de JavaScript, que nos permite editar los estilos, lo m\'etodos entre otras propiedades.
Mod\'ifica y no realiza ningun render nuevo.


\subsection{useMemo}

Es un hooks de React que te permite guardar en cach\'e el resultado de un c\'alculo entre renderizados. 

\begin{lstlisting}
    const cachedValue = useMemo(calculateValue, dependencies);
\end{lstlisting}

{\Large\bf Referencia}

Se llama a useMemo en el nivel superior de tu compoente para guardar en cach\'e un c\'alculo entre rerenderizados: 

{\normalsize\bf Par\'ametros}


\begin{itemize}
    \item[calcularValor: ] la funci\'on que calcula el valor que deseas memorizar. Debe ser pura, no debe acerptar argumentos y debe devolver un valor de cualquier tipo. React llamar\'a a tu funci\'on durante el renderizado inicial. 
    En renderizados posteriores, React devolver\'a el mismo valor nuevamente si las dependecias no han cambiado desde el \'utimo renderizado. De lo contrario, llamar\'a a calcularValor, devolver\'a su resultado y lo almacenar\'a en caso de que pueda reutilizarse m\'as tarde.
    \item[dependecias: ] la lista de todos los valores reactivos a los que se hace referencia dentro del c\'odigo calcularValor. Los valores reactivos incluyen props, estado y todas las variables y funciones declaradas directamente dentro del cuerpo de tu componente. Si tu linter
\end{itemize}


\subsection{useInterval}


\section{react-router-dom}

\subsection{Enrutamiento del lado del cliente}

React router, es una librería que nos permite actualizar la URL desde un clic sin realizar una solicitud
de otro documento desde el servidor, esto permite mostrar inmediatamente una nueva interfaz de usuario y realizar solicitudes
de datos con b\'usqueda para actualizar la p\'agina con nueva informaci\'on.

Esto permite experiencias de usuario m\'as r\'apidas porque el navegador no necesita solicitar un documento
completamente nuevo o volver a evaluar los activos de CSS y JavaScript para sigueinte p\'agina. Tambi\'en permite experiencias de usuario m\'as din\'amizas
con cosas como la animaci\'on.

\subsection{Rutas anidadas}
El enrutamiento anidado es la idea general de acoplar segmentos de URL a la jerarqu\'ia y los datos de los componentes. Las rutas anidadas de React
Router se inspiraron en el sistema de enrutamiento de Ember.js alrededor de 2014. El equipo de Ember se dio cuenta de que, en casi todos los casos, los segmentos de la URL determinan:
\begin{itemize}
  \item Los dise\~nos para presentar en la p\'agina.
  \item Las dependencias de datos de esos dise\~nos.
\end{itemize}

React Router adopta est\'a convenci\'on con API para crear dise\~nos anidados acoplados a segmentos de URL y datos.


\subsection{Segmentos din\'amicos}

Se utiliza el hooks useParams de React-Router-Dom el cu\'al se utiliza para capturar los parametros en la URL en los componentes,
por ejemplo,
\begin{verbatim}
  <Route  exact path="/usuario/:username/:age"  element={<Usuario />} />
\end{verbatim}
En la componente usuario si se invoca el useParams se obtiene un objeto
\begin{verbatim}
  {
    username:"Nombre_especificado_Url",
    age:"Edad_especificada"
  }
\end{verbatim}