\section{Hooks}
Los hooks son una nueva caracter\'istica en React 16.8. Estos me permiten usar el estado y otras caracteristicas de React sin escribir un clase.

Los hooks resuelven una amplia variedad de problemas aparentemente desconectados en React que hemos encontrado durante m\'as de conco a\~nos de escribir
mantener decenas de miles de componentes.

Esta secc\'on es tomada de \href{https://es.reactjs.org/docs/hooks-intro.html}{\blue Documentaci\'on de React}

En lo siguiente veremos unos tantos hooks.
\subsection{useState}
\begin{itemize}
    \item Use a state variable when a component needs to ``remember'' some information between renders.
    \item State variables are declared by calling the useState Hook.
    \item Hooks are special functions that start with use. They let you ``hook into'' React features like state.
    \item Hooks might remind you of imports: they need to be called unconditionally. Calling Hooks, including useState, is only valid at the top level of a component or another Hook.
    \item The useState Hook returns a pair of values: the current state and the function to update it.
    \item You can have more than one state variable. Internally, React matches them up by their order.
    \item State is private to the component. If you render it in two places, each copy gets its own state.
\end{itemize}

\begin{verbatim}
    import { useState } from 'react';
    import { sculptureList } from './data.js';
    
    export default function Gallery() {
      const [index, setIndex] = useState(0);
      const [showMore, setShowMore] = useState(false);
    
      function handleNextClick() {
        setIndex(index + 1);
      }
    
      function handleMoreClick() {
        setShowMore(!showMore);
      }
    
      let sculpture = sculptureList[index];
      return (
        <>
          <button onClick={handleNextClick}>
            Next
          </button>
          <h2>
            <i>{sculpture.name} </i> 
            by {sculpture.artist}
          </h2>
          <h3>  
            ({index + 1} of {sculptureList.length})
          </h3>
          <button onClick={handleMoreClick}>
            {showMore ? 'Hide' : 'Show'} details
          </button>
          {showMore && <p>{sculpture.description}</p>}
          <img 
            src={sculpture.url} 
            alt={sculpture.alt}
          />
        </>
      );
    }
\end{verbatim}
\begin{verbatim}


  export const sculptureList = [{
  name: 'Homenaje a la Neurocirugía',
  artist: 'Marta Colvin Andrade',
  description: 'Although Colvin is predominantly known for 
                abstract themes that allude to pre-Hispanic symbols, 
                this gigantic sculpture, an homage to neurosurgery, 
                is one of her most recognizable public art pieces.',
  url: 'https://i.imgur.com/Mx7dA2Y.jpg',
  alt: 'A bronze statue of two crossed hands delicately holding a human
        brain in their fingertips.'  
}, {
  name: 'Floralis Genérica',
  artist: 'Eduardo Catalano',
  description: 'This enormous (75 ft. or 23m) silver flower is located in
                Buenos Aires. It is designed to move, closing its petals in 
                the evening or when strong winds blow and opening them 
                in the morning.',
  url: 'https://i.imgur.com/ZF6s192m.jpg',
  alt: 'A gigantic metallic flower sculpture with reflective 
        mirror-like petals and strong stamens.'
}, {
  name: 'Eternal Presence',
  artist: 'John Woodrow Wilson',
  description: 'Wilson was known for his preoccupation with equality, 
                social justice, as well as the essential and spiritual 
                qualities of humankind. This massive (7ft. or 2,13m) 
                bronze represents what he described as "a symbolic Black 
                presence infused with a sense of universal humanity."',
  url: 'https://i.imgur.com/aTtVpES.jpg',
  alt: 'The sculpture depicting a human head seems ever-present and solemn. 
        It radiates calm and serenity.'
}, {
  name: 'Moai',
  artist: 'Unknown Artist',
  description: 'Located on the Easter Island, there are 1,000 moai, or extant
                 monumental statues, created by the early Rapa Nui people, 
                 which some believe represented deified ancestors.',
  url: 'https://i.imgur.com/RCwLEoQm.jpg',
  alt: 'Three monumental stone busts with the heads that are disproportionately 
        large with somber faces.'
}, {
  name: 'Blue Nana',
  artist: 'Niki de Saint Phalle',
  description: 'The Nanas are triumphant creatures, symbols of femininity and 
                maternity. Initially, Saint Phalle used fabric and found 
                objects for the Nanas, and later on introduced polyester to 
                achieve a more vibrant effect.',
  url: 'https://i.imgur.com/Sd1AgUOm.jpg',
  alt: 'A large mosaic sculpture of a whimsical dancing female figure in a 
        colorful costume emanating joy.'
}, {
  name: 'Ultimate Form',
  artist: 'Barbara Hepworth',
  description: 'This abstract bronze sculpture is a part of The Family of Man 
                series located at Yorkshire Sculpture Park. Hepworth chose not to 
                create literal representations of the world but developed abstract
                 forms inspired by people and landscapes.',
  url: 'https://i.imgur.com/2heNQDcm.jpg',
  alt: 'A tall sculpture made of three elements stacked on each other reminding 
        of a human figure.'
}, {
  name: 'Cavaliere',
  artist: 'Lamidi Olonade Fakeye',
  description: "Descended from four generations of woodcarvers, Fakeye's work 
                blended traditional and contemporary Yoruba themes.",
  url: 'https://i.imgur.com/wIdGuZwm.png',
  alt: 'An intricate wood sculpture of a warrior with a focused face on a horse 
        adorned with patterns.'
}, {
  name: 'Big Bellies',
  artist: 'Alina Szapocznikow',
  description: "Szapocznikow is known for her sculptures of the fragmented body 
                as a metaphor for the fragility and impermanence of youth and
                 beauty. This sculpture depicts two very realistic large bellies 
                 stacked on top of each other, each around five feet (1,5m) tall.",
  url: 'https://i.imgur.com/AlHTAdDm.jpg',
  alt: 'The sculpture reminds a cascade of folds, quite different from bellies in
         classical sculptures.'
}, {
  name: 'Terracotta Army',
  artist: 'Unknown Artist',
  description: 'The Terracotta Army is a collection of terracotta sculptures 
                depicting the armies of Qin Shi Huang, the first Emperor of China. 
                The army consisted of more than 8,000 soldiers, 130 chariots with 
                520 horses, and 150 cavalry horses.',
  url: 'https://i.imgur.com/HMFmH6m.jpg',
  alt: '12 terracotta sculptures of solemn warriors, each with a unique facial
        expression and armor.'
}, {
  name: 'Lunar Landscape',
  artist: 'Louise Nevelson',
  description: 'Nevelson was known for scavenging objects from New York City debris, 
                which she would later assemble into monumental constructions. 
                In this one, she used disparate parts like a bedpost, juggling pin,
                 and seat fragment, nailing and gluing them into boxes that reflect 
                 the influence of Cubism's geometric abstraction of space and form.',
  url: 'https://i.imgur.com/rN7hY6om.jpg',
  alt: 'A black matte sculpture where the individual elements are initially 
        indistinguishable.'
}, {
  name: 'Aureole',
  artist: 'Ranjani Shettar',
  description: 'Shettar merges the traditional and the modern, the natural and the 
                industrial. Her art focuses on the relationship between man and 
                nature. Her work was described as compelling both abstractly and 
                figuratively, gravity defying, and a "fine synthesis of unlikely 
                materials."',
  url: 'https://i.imgur.com/okTpbHhm.jpg',
  alt: 'A pale wire-like sculpture mounted on concrete wall and descending on the 
        floor. It appears light.'
}, {
  name: 'Hippos',
  artist: 'Taipei Zoo',
  description: 'The Taipei Zoo commissioned a Hippo Square featuring submerged 
                hippos at play.',
  url: 'https://i.imgur.com/6o5Vuyu.jpg',
  alt: 'A group of bronze hippo sculptures emerging from the sett sidewalk as if 
        they were swimming.'
}];

    \end{verbatim} 
\subsection{useContext}
\href{https://dmitripavlutin.com/react-context-and-usecontext/}{useContext}

\subsection{useRef}

useRef es un React hook que te permite hacer referencia a un valor que no es necesario para renderizar.

Le aplica una referencia a los elemetos del DOM de JavaScript, que nos permite editar los estilos, lo m\'etodos entre otras propiedades.
Mod\'ifica y no realiza ningun render nuevo.


\subsection{useMemo}

Es un hooks de React que te permite guardar en cach\'e el resultado de un c\'alculo entre renderizados. 

\begin{lstlisting}
    const cachedValue = useMemo(calculateValue, dependencies);
\end{lstlisting}

{\Large\bf Referencia}

Se llama a useMemo en el nivel superior de tu compoente para guardar en cach\'e un c\'alculo entre rerenderizados: 

{\normalsize\bf Par\'ametros}


\begin{itemize}
    \item[calcularValor: ] la funci\'on que calcula el valor que deseas memorizar. Debe ser pura, no debe acerptar argumentos y debe devolver un valor de cualquier tipo. React llamar\'a a tu funci\'on durante el renderizado inicial. 
    En renderizados posteriores, React devolver\'a el mismo valor nuevamente si las dependecias no han cambiado desde el \'utimo renderizado. De lo contrario, llamar\'a a calcularValor, devolver\'a su resultado y lo almacenar\'a en caso de que pueda reutilizarse m\'as tarde.
    \item[dependecias: ] la lista de todos los valores reactivos a los que se hace referencia dentro del c\'odigo calcularValor. Los valores reactivos incluyen props, estado y todas las variables y funciones declaradas directamente dentro del cuerpo de tu componente. Si tu linter
\end{itemize}


\subsection{useInterval}


\section{react-router-dom}

\subsection{Enrutamiento del lado del cliente}

React router, es una librería que nos permite actualizar la URL desde un clic sin realizar una solicitud
de otro documento desde el servidor, esto permite mostrar inmediatamente una nueva interfaz de usuario y realizar solicitudes
de datos con b\'usqueda para actualizar la p\'agina con nueva informaci\'on.

Esto permite experiencias de usuario m\'as r\'apidas porque el navegador no necesita solicitar un documento
completamente nuevo o volver a evaluar los activos de CSS y JavaScript para sigueinte p\'agina. Tambi\'en permite experiencias de usuario m\'as din\'amizas
con cosas como la animaci\'on.

\subsection{Rutas anidadas}
El enrutamiento anidado es la idea general de acoplar segmentos de URL a la jerarqu\'ia y los datos de los componentes. Las rutas anidadas de React
Router se inspiraron en el sistema de enrutamiento de Ember.js alrededor de 2014. El equipo de Ember se dio cuenta de que, en casi todos los casos, los segmentos de la URL determinan:
\begin{itemize}
  \item Los dise\~nos para presentar en la p\'agina.
  \item Las dependencias de datos de esos dise\~nos.
\end{itemize}

React Router adopta est\'a convenci\'on con API para crear dise\~nos anidados acoplados a segmentos de URL y datos.


\subsection{Segmentos din\'amicos}

Se utiliza el hooks useParams de React-Router-Dom el cu\'al se utiliza para capturar los parametros en la URL en los componentes,
por ejemplo,
\begin{verbatim}
  <Route  exact path="/usuario/:username/:age"  element={<Usuario />} />
\end{verbatim}
En la componente usuario si se invoca el useParams se obtiene un objeto
\begin{verbatim}
  {
    username:"Nombre_especificado_Url",
    age:"Edad_especificada"
  }
\end{verbatim}