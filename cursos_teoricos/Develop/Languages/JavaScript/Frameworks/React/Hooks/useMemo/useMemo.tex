\subsection{useMemo}

Es un hooks de React que te permite guardar en cach\'e el resultado de un c\'alculo entre renderizados. 

\begin{lstlisting}
    const cachedValue = useMemo(calculateValue, dependencies);
\end{lstlisting}

{\Large\bf Referencia}

Se llama a useMemo en el nivel superior de tu compoente para guardar en cach\'e un c\'alculo entre rerenderizados: 

{\normalsize\bf Par\'ametros}


\begin{itemize}
    \item[calcularValor: ] la funci\'on que calcula el valor que deseas memorizar. Debe ser pura, no debe acerptar argumentos y debe devolver un valor de cualquier tipo. React llamar\'a a tu funci\'on durante el renderizado inicial. 
    En renderizados posteriores, React devolver\'a el mismo valor nuevamente si las dependecias no han cambiado desde el \'utimo renderizado. De lo contrario, llamar\'a a calcularValor, devolver\'a su resultado y lo almacenar\'a en caso de que pueda reutilizarse m\'as tarde.
    \item[dependecias: ] la lista de todos los valores reactivos a los que se hace referencia dentro del c\'odigo calcularValor. Los valores reactivos incluyen props, estado y todas las variables y funciones declaradas directamente dentro del cuerpo de tu componente. Si tu linter
\end{itemize}

