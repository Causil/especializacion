\part{Python}

\chapter{Instalaciones en Python}

Every project gets its own \href{https://docs.python.org/3/library/venv.html}{\blue venv}, too. Simply type ``python3 -m venv ./venv'' in your
terminal (or ``python'' or whatever the name of your executable is) in your development
folder. I won't upload my virtual environment to GitHub of course, so you have to do this
on your own.

Open project folder in VSCode \& download data
Then select the new venv python-executable as the current project's python interpreter
and then every new terminal in VSCode should automatically have the virtual
environment activated. Create a jupyter notebook file (mine is called
``explore\_data.ipynb'' for now) and select the new venv as kernel. It will automatically
suggest to install ipykernel and all that when you execute the first cell.Then I download
the Kaggle dataset using the Kaggle API. After you registered at Kaggle, you can use
the command

Nota para instalar kaggle primero se debe instalar de forma general fuera del entorno virtual y despu\'es de forma local en en el 
entorno virtual, y darle permisos con el comando chmod 600 $\sim$/.kaggle/kaggle.json, para instalarlo de forma general se utiliza el comando pip install --user kaggle, se dan los permisos
utilizando chmod 600 $\sim$/.kaggle/kaggle.json, se tiene que inscirbir a la competici\'on en kaggle para poder participar en la competici\'on y verificar por medio de un c\'odigo en el tel\'efono



\chapter{An\'alisis de datos}

\section{Tricks for Data Science \& Data Analysis}
\begin{enumerate}
	\item Remover duplicados de una lista: no utilizar for mejor utilizar set
	     \begin{verbatim}
	     	cities = ["Abidjan","Bamako","Dakar","Adibjan", "Dakar"]
	     	Don't do this 
	     	Firts Approach
	     	
	     	for city in cities:
	     		if city not in unique_cities:
	     			unique_cities.append(city)
	     	print(unique_cities) -> ["Abidjan","Bamako","Dakar"]
	     	
	     	Do this
	     	Second Aproach
	     	
	     	unique_cities = list(set(sities))
	     	print(unique_cities)
	     \end{verbatim}
\end{enumerate}

\subsection{Anaconda}
Contiene un grupo de librer\'ias de python que son Numpy, Scipy, matplotlib, python, pandas, jupyter, scikitLearn, Spyder y para instalar librer\'ias adicionales trae un gestor de paquetes que llama {\bf conda}, podemos hacer la instalaci\'on de keras, TensorFlow, PyTorch.

\url{https://www.anaconda.com/products/distribution}

V\'ideo explicando como instalar anaconda.
\url{https://www.youtube.com/watch?v=kymjQ6e4jsQ}
\section{Instalando Python}

\section{instalando TensorFlow}

Para instalar TensorFlow nos podemos guiar de \url{https://www.tensorflow.org/install/pip}

Si ya tienes el entorno virtual instalado y activado solo es ejecutar en el entorno virtual 
\begin{verbatim}
	pip install --upgrade pip
\end{verbatim}



\section{Instalando Jupyter Notebook}
Ejecutar Jupyter dentro del entorno virtual, 

\url{https://es.acervolima.com/uso-de-jupyter-notebook-en-un-entorno-virtual/}

Una vez creado el entorno virtual ejecutamos 

abrimos el Jupyter Notebook y procedemos a cambiar el kernel en la opci\'on venv que es como se llamo nuestro kernel, de esta manera podemos acceder a todos los paquetes instalados en nuestro entorno virtual. Para desactivarlo procedemos a ejecutar el siguiente comando: 
\begin{verbatim}
	jupyter-kernelspec uninstall venv
\end{verbatim}
Esto fue consultado en \cite{JupyterEnviroments}.
En si solo es cambiar la direcci\'on del kernel.

Pendiente hacer esto mismo en vscode

\chapter{Numpy}

\section{Dimensiones en matrices}

\subsection{0-D Arrays}
0-D arrays, or scalars, are the elements in a array. Each value in an array is a 0-D array.

\begin{verbatim}
	import numpy as np
	
	arr = np.array(42)
	
	print(arr) 
	
	ouput()
			42
\end{verbatim}


\subsection{1-D Arrays}
An array that has 0-D arrays as its elements is called uni-dimensional or 1-D array.

\begin{verbatim}
Example

Create a 1-D array containing the values 1,2,3,4,5:
import numpy as np

arr = np.array([1, 2, 3, 4, 5])

print(arr) 
print(type(arr)) 
output()
   [1 2 3 4 5]
   <class 'numpy.ndarray'>
\end{verbatim}
No confundir con una lista de Python son totalmente diferentes.


\subsection{2-D Arrays}
An array that has 1-D arrays as its elements is called a 2-D array.

These are often used to represent matrix or 2nd order tensors.

NumPy has a whole sub module dedicated towards matrix operations called numpy.mat

\begin{verbatim}
Example

Create a 2-D array containing two arrays with the values 1,2,3 and 4,5,6:
import numpy as np

arr = np.array([[1, 2, 3], [4, 5, 6]])

print(arr) 
\end{verbatim}




\subsection{3-D Arrays}
An array that has 2-D arrays (matrices) as its elements is called 3-D array.

These are often used to represent a 3rd order tensor.

\begin{verbatim}
Example

Create a 3-D array with two 2-D arrays, both containing two arrays with the values 1,2,3 and 4,5,6:
import numpy as np

arr = np.array([[[1, 2, 3], [4, 5, 6]], [[1, 2, 3], [4, 5, 6]]])

print(arr) 

output()
[ [
  [1 2 3]
  [4 5 6]
  ]
  [
  [1 2 3]
  [4 5 6]
  ]
]
\end{verbatim}


Check Number of Dimensions?

NumPy Arrays provides the ndim attribute that returns an integer that tells us how many dimensions the array have.


\begin{verbatim}
	Example
	
	Check how many dimensions the arrays have:
	import numpy as np
	
	a = np.array(42)
	b = np.array([1, 2, 3, 4, 5])
	c = np.array([[1, 2, 3], [4, 5, 6]])
	d = np.array([[[1, 2, 3], [4, 5, 6]], [[1, 2, 3], [4, 5, 6]]])
	
	print(a.ndim)
	print(b.ndim)
	print(c.ndim)
	print(d.ndim) 
	output()
	0
	1
	2
	3
\end{verbatim}

Higher Dimensional Arrays

An array can have any number of dimensions.

When the array is created, you can define the number of dimensions by using the ndmin argument.

\begin{verbatim}
	Example
	
	Create an array with 5 dimensions and verify that it has 5 dimensions:
	import numpy as np
	
	arr = np.array([1, 2, 3, 4], ndmin=5)
	
	print(arr)
	print('number of dimensions :', arr.ndim) 
	
	output()
	[[[[[1 2 3 4]]]]]
	number of dimensions : 5
\end{verbatim}
In this array the innermost dimension (5th dim) has 4 elements, the 4th dim has 1 element that is the vector, the 3rd dim has 1 element that is the matrix with the vector, the 2nd dim has 1 element that is 3D array and 1st dim has 1 element that is a 4D array.

\section{Numpy data types}

Data Types in Python

By default Python have these data types:

strings - used to represent text data, the text is given under quote marks. e.g. ``ABCD''


integer - used to represent integer numbers. e.g. -1, -2, -3

float - used to represent real numbers. e.g. 1.2, 42.42

boolean - used to represent True or False.

complex - used to represent complex numbers. e.g. 1.0 + 2.0j, 1.5 + 2.5j

Data Types in NumPy

NumPy has some extra data types, and refer to data types with one character, like i for integers, u for unsigned integers etc.

Below is a list of all data types in NumPy and the characters used to represent them.

i - integer\\
b - boolean\\
u - unsigned integer\\
f - float\\
c - complex float\\
m - timedelta\\
M - datetime\\
O - object\\
S - string\\
U - unicode string\\
V - fixed chunk of memory for other type ( void )\\


Shape of an Array

The shape of an array is the number of elements in each dimension.

Get the Shape of an Array

NumPy arrays have an attribute called shape that returns a tuple with each index having the number of corresponding elements.

The shape arrays is return tupla where, first element is 

L aforma de un arreglo 


Estudiando de 
\url{https://www.w3schools.com/python/numpy/numpy_array_join.asp} y python para Data Science

\section{Shape de un arreglo}
Un array o arreglo de numpy tiene un atributo llamado shape que retorna una tupla donde cada indice tiene el numero correspondiente de elementos, ejemplo: 
\begin{verbatim}
	Example
	
	Print the shape of a 2-D array:
	import numpy as np
	
	arr = np.array([[1, 2, 3, 4], [5, 6, 7, 8]])
	
	print(arr.shape) 
	
	output: 
	
	(2,4)
	
	Nos \'indica que tenemos un arreglo de dos elementos de una
	dimensi\'on y cada elemenos de una dimensi\'on tiene cuatro elementos.
	
	import numpy as np
	
	arr = np.array([1, 2, 3, 4], ndmin=5)
	
	print(arr)
	print('shape of array :', arr.shape) 
	import numpy as np
	
	arr = np.array([1, 2, 3, 4], ndmin=5)
	
	print(arr)
	
	print('shape of array :', arr.shape)
	
	[[[[[1 2 3 4]]]]]
	shape of array : (1, 1, 1, 1, 4)
\end{verbatim}

\section{NumPy Array Reshaping}

Reshaping o reformar un array o arreglo significa cambiar la forma de un array, el shape de un arreglo es el n\'umero de elementos en cada dimenci\'on. Al reshaping podemos agregar o eliminar dimensiones o cambiar el n\'umero de elementos en cada dimensi\'on. 

\subsection{Reshape de 1-D a 2-D}

Ejemplo: 

Convirtamos el siguiente array de  1-D con 12 elementos en un arreglo de 2-D, donde la dimensi\'on 2-D tendra 4 elementos y cada elemento de una dimensi\'on tendra 3 elementos.

\begin{verbatim}
	import numpy as np
	
	arr = np.array([1, 2, 3, 4, 5, 6, 7, 8, 9, 10, 11, 12])
	
	newarr = arr.reshape(4, 3)
	
	print(newarr)
	
	output:
	[[ 1  2  3]
	[ 4  5  6]
	[ 7  8  9]
	[10 11 12]]
\end{verbatim}

Ahora convirtamos el arreglo de 1-D en un arreglo en 3-D, es decir: 

\begin{verbatim}
	import numpy as np
	
	arr = np.array([1, 2, 3, 4, 5, 6, 7, 8, 9, 10, 11, 12])
	
	newarr = arr.reshape(2, 3, 2)
	
	print(newarr) 
	output:
	
	import numpy as np
	
	arr = np.array([1, 2, 3, 4, 5, 6, 7, 8, 9, 10, 11, 12])
	
	newarr = arr.reshape(2, 3, 2)
	
	print(newarr)
	
	[[[ 1  2]
	[ 3  4]
	[ 5  6]]
	
	[[ 7  8]
	[ 9 10]
	[11 12]]]
\end{verbatim}

El elemento retornado tiene la misma direcci\'on de memoria el arreglo original, es decir sucede lo siguiente:
\begin{verbatim}
	import numpy as np
	
	arr = np.array([1, 2, 3, 4, 5, 6, 7, 8, 9, 10, 11, 12])
	
	newarr = arr.reshape(4, 3)
	newarr[1][0]=2
	print(newarr)
	print(arr)
	output: 
	[[ 1  2  3]
	[ 2  5  6]
	[ 7  8  9]
	[10 11 12]]
	
	[ 1  2  3  2  5  6  7  8  9 10 11 12]
\end{verbatim}

Dimensi\'on desconocida

Se le permite tener una dimensi\'on desconocida.

Lo que significa que no tiene que especificar un n\'umero exacto para una de las dimensiones en el m\'etodo de reshape.

Pasa -1 como valor y numpy calcular\'a este n\'umero por usted.

\begin{verbatim}
	Example
	
	Convert 1D array with 8 elements to 3D array with 2x2 elements:
	import numpy as np
	
	arr = np.array([1, 2, 3, 4, 5, 6, 7, 8])
	
	newarr = arr.reshape(2, 2, -1)
	
	print(newarr) 
	[[[1 2]
	[3 4]]
	
	[[5 6]
	[7 8]]]
\end{verbatim}
Note: We can not pass -1 to more than one dimension.

Aplanando los arreglos

Aplanar un arreglo multidimensional significa convertir una matriz multidimensional en una matriz 1D.

podemos usar reshape(-1)

\begin{verbatim}
	Example
	
	Convert the array into a 1D array:
	import numpy as np
	
	arr = np.array([[1, 2, 3], [4, 5, 6]])
	
	newarr = arr.reshape(-1)
	
	print(newarr) 
	output:
	
	[1 2 3 4 5 6] 
\end{verbatim}
\section{M\'etodos}


