\part{Bases de Datos}
\chapter{Relaciones}

Objetos 

Entidad rectangulos

\chapter{MySQL}

\section{Funciones de control de flujo}
\begin{description}
	\item[Case: ] Esta expresi\'on nos permite realizar la condici\'on y devolver el primer valor que cumpla con dicha condici\'on
		\begin{example}
			\begin{verbatim}
				
			Primer ejemplo
			
				select 
					case 1
					when 1 then 'uno'
					when 2 then 'dos'
					else 'otro n\'umero'
					end as valor;
					
			segundo ejemplo:
			
				select idFactura, idProducto,
				
				case 
					when cantidad > 2 then 'M\'as de dos productos vendidos'
					when cantidad = 2 then 'Dos productos vendidos'
					else 'Menos de dos productos vendidos'
				end as cantidad
				from detalle_factura;

			Tercer ejemplo
			
			select nombre,
			case
				when email IS NULL then 'No tiene email registrado'
				else 'email'
			end as email,
			pais
			from cliente;
			\end{verbatim}
		Podemos ver que es una sentencia muy similar a switch de vuelve el primer caso que cumpla la condici\'on. 
		\end{example}
	\item[IF :] 
	\begin{example}
		\begin{verbatim}

			Primer ejemplo
			select if(1 < 2, true, false) as resultado;
			
			segundo ejemplo
			
			select 
			idProducto,
			if(cantidad > 1, cantidad*precioUnitario, precioUnitario) as total
			from detalle_factura;
			
			Tercer ejemplo
			
			select
			nombre,
			if(fechaIngreso < '2016-12-31', concat(idEmpleado, '-16'),
				if(fechaIngreso < '2017-12-31', concat(idEmpleado, '-17'),
					if(fechaIngreso < '2018-12-31', concat(idEmpleado, '-18'),
						concat(idEmpleado,'-19')
					  )
				   )
			   ) as codigo
			from empleado;
			 
		\end{verbatim}
	\end{example}
	\item[IFNULL y NULLIF:] IFNULL nos permite evaluar una primera expresi\'on, si esta expresi\'on es null, entonces devolver\'a el segundo valor pasado por par\'ametro y NULLIF : 
	\begin{example}
		\begin{verbatim}
			
			Primer ejemplo: 
			
			select ifnull(null, 'texto') as resultado;
			
			Segundo ejemplo: 
			
			En este ejemplo devuelve los contactos de la tabla cliente en la
			columna nombre si tiene email nos da el email pero si este campo
			es null nos devuelve el tel\'efono
			
			select nombre, ifnull(email, telefono) as contacto
			from cliente;
			
			Tercer ejemplo: 
			
			select nombre,
			ifnull( (select email from cliente where idCliente = '14'), 
			'No tiene eamil registrado' )
			as email 
			from cliente
			where idCliente = '14';
			
			select 
			nullif(
				(select precioUnitario from producto where idProducto = 1),
				(select )
			)
		\end{verbatim}
	\end{example}
	
	\item[NULLIF:]

\end{description}

\section{Subqueries}
Es una declaraci\'on select en otra declaraci\'on, los subqueries devuelven datos de la consulta principal, los subqueries puede ser utilizados para agregar, actualizar, eliminar, enviar datos.

\begin{example}
	\begin{verbatim}
		
	 Ejemplo n\'umero 1: 
	 Consiste en traer cuyos empleados tengan mayor salario al promedio: 
	 
	 select idEmpleado, nombre, salario
	 from empleado
	 where salario > (select avg(salario) from empleado);

	Ejemplo 2: Seleccionamos los empleados que no pertenezcan al departamento general: 
	
	select nombre, apelllido, idDepartamento
	from empleado
	where idDepartamento NOT IN (select idDepartamento 
	                             from departamento
	                             where nombre like "%general%"
	                              );
	Ejemplo 3: facturas de los clientes que pertenezacan a Canada o Brasil:
	
	select idCliente, idFactura
	from factura
	where idCliente IN( select idCliente 
						from cliente
						where pais = 'canada' or pais = 'Brasil'
						 );
	
	\end{verbatim}
\end{example}


Subconsultas: 

\begin{example}
	\begin{verbatim}
		select *
		from factura
		where idCliente = (select idCliente form cliente where nombre = 'Jordi');
		
		select *
		from producto 
		where precioUnitario <= 
		(select avg(precioUnitario) from producto where idCategoria = 5)
		and idCategoria = 5;
		
		
	\end{verbatim}
\end{example}

comparando subconsultas

Subconsultas: 

\begin{example}
	\begin{verbatim}
		
		select idProducto, nombre
		from producto
		where idProducto = ANY (select idProducto from detalle_factura);
		
		
	\end{verbatim}
\end{example}