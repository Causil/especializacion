%-------------------%DEF DE FUNCION COMPUESTA------------------%
\newcommand{\compcent}[1]{\vcenter{\hbox{$#1\circ$}}}
\newcommand{\comp}{\mathbin{\mathchoice
  {\compcent\scriptstyle}{\compcent\scriptstyle}
  {\compcent\scriptscriptstyle}{\compcent\scriptscriptstyle}}}
%-------------------%DEF DE FUNCION COMPUESTA%-----------------% 

%------------------------------nuevos comandos------------------%
\newcommand{\N}{{\ensuremath{\mathbb{N}}}}
\newcommand{\Z}{{\ensuremath{\mathbb{Z}}}}
\newcommand{\Q}{{\ensuremath{\mathbb{Q}}}}
\newcommand{\R}{{\ensuremath{\mathbb{R}}}}
\newcommand{\C}{{\ensuremath{\mathbb{C}}}}
\newcommand{\K}{{\ensuremath{\mathbb{K}}}}
\newcommand{\J}{{\ensuremath{\mathbb{J}}}}
\newcommand{\s}{{\ensuremath{\mathbb{S}}}}

\newcommand{\mca}{{\ensuremath{\mathcal{A}}}}
\newcommand{\mcb}{{\ensuremath{\mathcal{B}}}}
\newcommand{\mcc}{{\ensuremath{\mathcal{C}}}}
\newcommand{\mcs}{{\ensuremath{\mathcal{S}}}}
\newcommand{\mct}{{\ensuremath{\mathcal{T}}}}
\newcommand{\mcu}{{\ensuremath{\mathcal{U}}}}
\newcommand{\mcp}{{\ensuremath{\mathcal{P}}}}
\newcommand{\mcm}{{\ensuremath{\mathcal{M}}}}
\newcommand{\mcn}{{\ensuremath{\mathcal{N}}}}
\newcommand{\mcj}{{\ensuremath{\mathcal{J}}}}
\newcommand{\mcf}{{\ensuremath{\mathcal{F}}}}
\newcommand{\mcd}{{\ensuremath{\mathcal{D}}}}
\newcommand{\mcx}{{\ensuremath{\mathcal{X}}}}
\newcommand{\mce}{{\ensuremath{\mathcal{E}}}}

\newcommand{\scra}{{\ensuremath{\mathscr{A}}}}
\newcommand{\scrt}{{\ensuremath{\mathscr{T}}}}
\newcommand{\scru}{{\ensuremath{\mathscr{U}}}}
\newcommand{\scrp}{{\ensuremath{\mathscr{P}}}}
\newcommand{\scrv}{{\ensuremath{\mathscr{V}}}}
\newcommand{\scro}{{\ensuremath{\mathscr{O}}}}
\newcommand{\scrb}{{\ensuremath{\mathscr{B}}}}
\newcommand{\scrc}{{\ensuremath{\mathscr{C}}}}
\newcommand{\scrf}{{\ensuremath{\mathscr{F}}}}
\newcommand{\scrs}{{\ensuremath{\mathscr{S}}}}
\newcommand{\scri}{{\ensuremath{\mathscr{I}}}}
\newcommand{\scrj}{{\ensuremath{\mathscr{J}}}}
\newcommand{\scrl}{{\ensuremath{\mathscr{L}}}}
\newcommand{\scrm}{{\ensuremath{\mathscr{M}}}}
\newcommand{\scrn}{{\ensuremath{\mathscr{N}}}}
\newcommand{\scrk}{{\ensuremath{\mathscr{K}}}}
\newcommand{\scre}{{\ensuremath{\mathscr{E}}}}
\newcommand{\scrq}{{\ensuremath{\mathscr{Q}}}}

% Funci\'on salto del gradiente 
\newcommand{\lbt}{\llbracket}
\newcommand{\rbt}{\rrbracket}

% producto interior
\newcommand{\lca}{\langle}
\newcommand{\rca}{\rangle}

\newcommand{\ndiv}{\hspace{-4pt}\not|\hspace{2pt}}

%-----------------funcion restricci\'on---------------% 
\newcommand{\restr}[1]{\raisebox{-.5ex}{$|$}_{#1}}
\newcommand{\restrd}[2]{{#1}\raisebox{-.5ex}{$|$}_{#2}}
%---------------------------------------------------%
\newcommand{\cmark}{\ding{51}}
\newcommand{\xmark}{\ding{55}}
\DeclareMathOperator{\inte}{int}
\DeclareMathOperator{\vol}{vol}
\DeclareMathOperator{\supp}{supp}
\DeclareMathOperator{\ips}{\langle \cdot, \cdot\rangle}
\DeclareMathOperator{\norm}{\|\cdot\|}
\DeclareMathOperator{\re}{Re}
\DeclareMathOperator{\id}{id}
\DeclareMathOperator{\im}{Im}
\def\card{\mathop{\rm card}}
\def\diam{\mathop{\rm diam}}
\def\dist{\mathop{\rm dist}}
\def\Lim{\mathop{\rm lim}}
\def\Inf{\mathop{\rm inf}}
\def\Max{\mathop{\rm max}}
\def\Min{\mathop{\rm min}}
\def\Liminf{\mathop{\rm lim\,inf}}
\def\Limsup{\mathop{\rm lim\,sup}}


\newcommand{\cl}{\overline}

\DeclareRobustCommand{\rchi}{{\mathpalette\irchi\relax}}
\newcommand{\irchi}[2]{\raisebox{\depth}{$#1\chi$}} % inner command, used by \rchi
%--Programing notation derivate Lagrange ---%
%#1 it's a variable of function
%#2 it's a variable of degree function
%#3 it's a variable of function
%#4 it's a variable that expecific if def o not
\newcommand{\dvlagrange}[4]{
  \hspace{-4.7mm}
  \notblank{#4}{
      \ifstrequal{#1}{alpha}{$\alpha^{#2}(#3)$}{$#1^{#2}(#3)$}
    }{
    \ifnumcomp{#2}{>}{2}{
      \ifstrequal{#1}{alpha}{$\alpha^{#2}(#3)$}{$#1^{(#2)}(#3)$}
    }{
      \ifstrequal{#2}{2}{$#1^{\prime\prime}(#3)$}{$#1^{\prime}(#3)$}
    }
  }
  \hspace{-4.7mm}
  }
\newcommand{\dvleibniz}[4]{
  \notblank{#4}{
    $\dfrac{\mathnormal{d}^{#2}  #1}{\mathnormal{ d #3^{#2}}}$
  }{
  \ifnumcomp{#2}{>}{1}{
    $\dfrac{\mathnormal{d}^{#2}  #1}{\mathnormal{d} #3^{#2}}$
  }{
    $\dfrac{\mathnormal{d} #1}{\mathnormal{d #3}}$
  }
}
}

\newcommand{\dvnewton}[2]{
  \ifnumcomp{#2}{>}{1}{
    $\overset{..}{#1}$
  }{
    $\overset{.}{#1}$
  }
}
 % si no queremos que a�ada la palabra "Capitulo"
%\addcontentsline{toc}{chapter}{Agradecimientos} % si queremos que aparezca en el �ndice
%\markboth{AGRADECIMIENTOS}{Agradecimiento} % encabezado 
%\addcontentsline{toc}{section}{Resumen} % si queremos que aparezca en el �ndice
%\markboth{RESUMEN}{Resumen} % encabezado
%\usepackage{nopageno}
%\usepackage{fancyhdr}
%\pagestyle{fancy}
%\pagestyle{empty}

\newcommand{\tree}[2]{%
    \ifstrequal{#1}{a}%
        {$\sqrt{#2}$}%
        {\ifstrequal{#1}{b}{Hi}{\PackageError{tree}{Undefined option to tree command}{}}}%
        }