\section{ Four Basic Counting Principles }
\begin{definition}{\bf (Pairwise disjoint sets.)}
    Let $S$ be a set. A partition of $S$ is a collection $S_1, S_2,\ldots, S_m$ of subsets of $S$ such that each element of $S$
    is in exactly one of those subsets:
    \begin{align*}
        S_1\cup S_2\cup \cdots \cup S_m = S \quad \text{ and } \quad S_i\cap S_j = \emptyset \quad \text{ for all } i\not=j.
    \end{align*}
    Thus, the sets $S_1,S_2,\ldots,S_m$ are pairwise disjoint sets, and their union is $S$. The subsets $S_1,S_2,\ldots,S_m$ are called the parts of the partition.

    The numbers of objects of a set $S$ is denoted by $|S|$ and is sometimes called the size of $S$.
\end{definition}
\begin{example}
    \hspace{0mm}
    \begin{enumerate}
        \item $\Omega = \{ T, H \}$ where $T$ denotes tails and $H$ denotes heads. The set $\Omega$ is a partition of the set of all possible outcomes of a single coin toss.
              $$\Omega = \{T\} \cup \{H\} $$
    \end{enumerate}
\end{example}

\subsection{Addition Principle}

\begin{definition}{\bf (Addition Principle.)}
    Suppose that a set $S$ is partitioned into pairwise disjoint parts $S_1, S_2, \ldots , S_n$. The number the 
    of objects in $S$ can be determined by finding the number of objects un each of the parts, and adding the 
    number so obtained:
    \begin{align}
        |S| = |S_1| + |S_2| + \ldots + |S_n|.
    \end{align}
    If the sets $S_1,S_2,\ldots,S_m$ are allowed to overlap, then a more profound principle, the inclusion-exclusion principle, is needed.
\end{definition}

\begin{example}
    \hspace{0mm}
    \begin{enumerate}
        \item 
    \end{enumerate}
\end{example}

\subsection{Multiplication Principle}

\begin{definition}{\bf (Multiplication Principle.)}
    Let $S$ be a set of ordered pairs $(a,b)$ of objects, where the first object a comer from a set of size $p$
    , and for each choice of object $a$ there are $q$ choices for object $b$. Then the size of $S$ is $p\times q$:
    \begin{align}
        |S| = p\times q.
    \end{align}
    The mulplication principle is actually a consequence of the addition principle.

    A second useful formulation of the multiplication principle is as follows: If a first
    task has $p$ outcomes and, no matter what the outcome of the first task, a second task
    has $q$ outcomes, then the two tasks performed consecutively have $p \times q$ outcomes.
\end{definition}
\subsection{Principio de la adici\'on}
Si un suceso $S_{1}$  ocurre de $a$ maneras y un suceso $S_{2}$ ocurre de $b$ maneras entonces puede decirse que el n\'umero total de maneras en que puede
ocurrir $S_1$ o el suceso $S_2$ es $a+b$.


\begin{example}
    \hspace{0mm}
    \begin{enumerate}[{\rm 1}]
        \item {\bf Ejemplo 1:}  ?`Cu\'antos resultados diferentes se pueden obtener lanzando un dado o una moneda?
        Indetificamos que el experimento consiste en elegir que objeto lanzar, suponga:
        $$ S_1\colon = \text{ lanzar el dado } \text{ y } S_2 \colon = \text{ lanzar la moneda. }$$
        Si se elije el dado, entonces el n\'umero de resultados posibles es $\{1,2,3,4,5,6\}$ o 
        si elige la moneda el n\'umero de resultados posibles es $\{C,S\}$. 
        As\'i el n\'umero total de resultados posibles para el suceso $S_1$ o $S_2$ es: $$6+2=8$$.
        \item {\bf Ejemplo 2:}  ?`Cu\'antos resultados diferentes se pueden obtener lanzando un dado y una moneda?
        Considerando los mismo sucesos $S_1$ y $S_2$ del ejemplo anterior, el experimiento consiste en lanzar un dado y seguidamente lanzar una moneda.
        nuestro conjunto de resultados ser\'ia
        $$ \{ {(1,C), (1,S)}, (2,C), (2,S), \ldots, (6,C), (6,S) \}. $$
        As\'i la cantidad de sucesos es $6\cdot2 = 12.$
        
    \end{enumerate}
\end{example}

\subsection{Principio de la multiplicaci\'on}
\begin{definition}{\bf (Principio de la multiplicaci\'on)}
    Si un experimento consiste de $n$ etapas o sucesos y si la primera etapa o suceso puede ser realizada de $n_{1}$ maneras, la segunda
    etapa de $n_{2}$ maneras, la $k$-\'esima etapa de $n_{k}$ maneras, entonces el experimento completo puede ser realizado
    de $n_{1}\cdot n_{2}\cdot\ldots\cdot n_{k}$ maneras.
\end{definition}


\subsection{Permutaciones}

?`De cu\'antas formas se pueden ordenar u organizar los elementos de un conjunto o de un subconjunto? Para responder tal pregunta, realicemos
las siguientes definiciones. 
\begin{definition}
    \hspace{0.5mm}
    \begin{enumerate}[{\rm 1.}]        
        \item {\rm Ordenar: } significa poner una cantidad de elementos en un orden espec\'ifico que obedece a una regla l\'ogica.
        \item {\rm Organizar: } implica poner una cantidad de elementos de forma que no tienen un orden espec\'ifico o dicho orden no es importante 
        dentro de un determinado contexto.
        \item {\rm Conjunto: } 
        \item {\rm Subconjunto: } 
        \item {\rm Orden: } corresponde a la aplicaci\'on de un criterio espec\'ifico que le confiere cierta jerarqu\'ia a los elementos que forman parte de un conjunto. 
        
    \end{enumerate}
\end{definition}

Definimos lso conceptos de permutaci\'on y combinaci\'on: 

\begin{definition}{Permuatci\'on}

\end{definition}

%\subsection{Talleres}
% \begin{enumerate}[{\rm 1.}]
%     \item Se selecciona al azar un veh\'iculo en cierta ciudad. Si todas las letras de la placa el veh\'iculo son diferentes, 
%     ?`Cu\'antos autos tienen la misma caracter\'istica? ?`Cu\'antos autos tienen placas con todos sus d\'igitos impares? \\
%     {\bf Soluci\'on:} 
%     Suceso:\\ $S_1\colon=$ formar una placa de auto eligiendo una letra del abcdario sin repetir y elegir un n\'umeros del $1$ al $10$.

    
%     Sabemos que las letras del abcedario son 27, aplicando el principio de la multiplicaci\'on, el n\'umero de placas posibles es:
%     $$ 27\cdot26\cdot25\cdot10\cdot10\cdot10= 17550000 $$

%     $$ 27P3\cdot10C3 = $$
%     \item {\bf Ejemplo 2:} Se lanzan $20$ monedas no cargadas, ?`Cu\'antos posibles resultados tienen solo tres caras?\\[5pt]
%     {\bf Soluci\'on:} 
%     $$\binom{20}{3} = \frac{20!}{(20-3)!3!} = \frac{20\cdot19\cdot18\cdot17!}{17!3!}=20\cdot19\cdot3=1140 $$
%     \item Se seleccionan al azar tres personas de un grupo por 10
%     obreros, 4 pintores y 6 carpinteros. ?`Cu\'antos grupos diferentes conformados por un obrero, un pintor y un carpintero se pueden formar?\\[5pt]
%     {\rm\bf Soluci\'on: } 
%     $$ 10\cdot4\cdot 6 = 240 $$
%     \item El pedido de una computadora personal digital puede especificar uno de cinco tama\~nos de memor\'ia, cualquier tipo de monitos de tres posibles,
%     cualquier tama\~no de disco duro de entre cuatro posibles, y puede incluir o no una tableta para l\'apiz electr\'onico. ?`Cu\'antos sistemas distintos pueden ordenarse?\\[5pt]
%     {\rm\bf Soluci\'on: }
%     $$ 5\cdot3\cdot4\cdot 2 = 120. $$
%     El total de sistemas que pueden formarse son $120$.
%     \item Un proceso de manufactura est\'a formado por $10$ operaciones, las cuales pueden efectuarse en cualquier orden. ?`Cu\'antas secuencias de producci\'on distintas son posibles?\\[5pt]
%     {\rm\bf Soluci\'on: }
%     $$ 10!=3628800  $$
%     \item Un proceso de manufactura est\'a formado por $10$ operaciones. Sin embargo, cinco de ellas deben terminarse antes de que pueda 
%     darse inicio a las otras cinco. Dentro de cada conjunto de cinco, las operaciones pueden efectuarse en cualquier orden. ?`Cu\'al es el n\'umero de secuencias de operaciones distintas posible?\\[5pt]
%     {\rm\bf Soluci\'on: }
%     $$ P_5^{10} +  =  $$
% \end{enumerate}