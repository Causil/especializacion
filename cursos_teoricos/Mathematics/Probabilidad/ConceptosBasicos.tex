\subsection{Conceptos b\'asicos}
\subsubsection{Espacio de probabilidad}

En est\'a secci\'on desarrollamos la noci\'on de medida de probabilidad y se presentan sus propiedades b\'asicas.

\begin{definition}{\bf (Experimiento aleatorio)}
Un experimento es aleatorio si su resultado no puede ser determinado de antemano.
\end{definition}

Si el conjunto de los posibles resultados de un experimento aleatorio es conocido, a este conjunto se le llamada espacio aleatorio

\begin{definition}{\bf (Espacio muestral)}
    El conjunto $\Omega$ de todos los posibles resultados de un experimiento aleatorio es llamado espacio muestral. Un elemento $w\in \Omega$ es llamado 
    un resultado o punto muestral.
\end{definition}

\begin{example}{\bf (Ejemplos de espacios muestrales)}
    \hspace{1cm}
    \begin{enumerate}[{\bf 1.}]
    \item {\bf Experimento:} Lanzar una moneda. Los posibles resultados en este caso son ``cara'' y ``sello''. Esto es 
    $$\Omega = \{ C,S \}$$
    \item {\bf Experimeinto:} Lanzar un dado ordianrio tres veces consecutivas. En este caso los posibles resultados son tripletas de la forma
    $(i,j,k)$ donde $i,j,k\in \{1,2,3,4,5,6\}$. Esto es
    $$\Omega = \{ (i,j,k) : i,j,k\in \{1,2,3,4,5,6\} \}$$
    \item {\bf Experimento:}  
    \end{enumerate}
\end{example}

\begin{definition}{\bf (Espacio muestral discreto)}
    Un espacio muestral $\Omega$, es llamado discreto si es finito o contable.
\end{definition}


\begin{definition}{\bf ($\sigma$ - \'Algebra)}
    Sea $\Omega\neq\emptyset$. Una colecci\'on $\mathcal{F}$ de subconjuntos de $\Omega$ es llamada $\sigma$-\'algebra si satisface las siguientes propiedades:
    \begin{enumerate}
        \item $\Omega\in\mathcal{F}$.
        \item Si $A\in\mathcal{F}$, entonces $A^{c}\in\mathcal{F}$.
        \item Si $A_{1},A_{2},\ldots\in\mathcal{F}$, entonces $\bigcup{i=1}^{\infty}A_{i}\in\mathcal{F}$.
    \end{enumerate}
\end{definition}


