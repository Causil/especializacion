\chapter{Introducci\'on a las series temporales}
Para estudiar est\'a parte del libro nos fijaremos en el libro \cite{Pena2010}, para estudiar las series temporales.
\begin{definition} {\rm (\cite{Pena2010}) }
Una serie temporal es el resultado de observar los valores de una variable a lo largo del tiempo 
en intervalos regulares (cada d\'ia, cada mes, cada a\~no, etc.)
\end{definition}

Series estables en el tiempo o estacionarias.
\begin{enumerate}
    \item Oscilan alredor de un nivel constante.
    \item Su gr\'afico no muestra ninguna tendencia clara a crecer o decrecer en el tiempo.
\end{enumerate}

Series no estables en el tiempo o no estacionarias.
\begin{enumerate}
    \item La tendencia en el tiempo es variable, en sentido creciente o decreciente.
    \item Presentan una tendencia evolutiva o cambiante en el tiempo.
\end{enumerate}

Series estacionales
\begin{enumerate}
    \item El valor medio de la variable observada dependa del intervalo seleccionado.
\end{enumerate}

Cu\'ando el nivel de la serie no es estable decimos que la serie no es estacionaria.

\newpage
Diccionario
\begin{description}
    \item[Fluctuaciones: ] se refieren a los cambios, variaciones o desviaciones que experimenta 
    una cantidad o un valor a lo largo del tiempo, de forma que no sigue un patr\'on 
    completamente predecible. 
    En otras palabras, las fluctuaciones son las oscilaciones o movimientos alrededor de un 
    nivel o valor promedio, ya sea en una serie temporal o en cualquier sistema din\'amico.
    \begin{enumerate}
        \item Naturaleza impredecible: Las fluctuaciones pueden ocurrir de manera aleatoria o debido a factores externos que influyen en el sistema.
        \item Magnitud variable: Las fluctuaciones pueden ser peque\~nas y casi imperceptibles, o pueden ser grandes y representar cambios significativos en los valores de la serie.
        \item Direcci\'on: Las fluctuaciones pueden ser hacia arriba o hacia abajo, es decir, pueden reflejar aumentos o disminuciones en el valor observado.
    \end{enumerate}
    Tipos de fluctuaciones: 
    \begin{enumerate}
        \item Fluctuaciones aleatorias(ruido): Son cambios que ocurren debido a factores impredecibles y no siguen ning\'un patr\'on espec\'ifico. Se consideran ruido en los datos.
        \item Fluctuaciones c\'iclicas o estacionales: En algunos sistemas, las fluctuaciones pueden seguir patrones repetitivos y predecibles, como los ciclos estacionales (verano/invierno), ciclos econ\'omicos, o patrones de demanda.
    \end{enumerate}
    \item[Efectos externos o efectos de intervenci\'on: ] 
\end{description}
