
\part{Mathematics}

A rules for understading mathematics
\begin{description}
    \item[Demostraci\'on: ] (del lat\'in demostratio-onem) Prueba de una verdad por medio del raciocinio, partiendo de principios evidentes. En ciencia, en general se puede extender a la comprobaci\'on experimental de un principio o teor\'ia.
    \item[Proposici\'on (del lat\'in propositio-onem): ] Enunciaci\'on de una verdad ya demostrada o que se ha de demostrar.
    \item[Silogismo (del lat\'in syllogismus, y \'este del griego): ] Argumento formado por tres proposiciones: premisa mayor, premisa menor y conclusi\'on. 
    \item[Enunciado (del lat\'in enuntiatio-onem)] Expresi\'on o manifestaci\'on de una idea.
    \item[Hip\'otesis (del lat\'in hypothesis, y \'este del griego): ] Suposici\'on de una idea, con el objetivo de deducir de ella alguna consecuencia.
    \item[Lema (del lat\'in lemma, y \'este del griego): ]      Proposici\'on cuya demostraci\'on antecede a un teorema.
    \item[Axioma (del lat\'in axioma, y \'este del griego): ]      Principio, verdad, sentencia clara y evidente, que no necesita demostraci\'on.
    \item[Conclusi\'on (del lat\'in conclusio-onem): ] Proposici\'on que se deduce o es consecuencia de las premisas.
    \item[Conjetura (del lat\'in conjectura): ] Opini\'on fundada en probabilidades, indicios o apariencias.
    \item[ Corolario (del lat\'in corollarium)] Proposici\'on que por s\'i sola se deduce de lo ya demostrado.
    \item[Premisa (del lat\'in praemissa): ]      Cualquiera de las dos proposiciones l\'ogicas de un silogismo, de donde se deduce la conclusi\'on. En un silogismo, a la proposici\'on más general se denomina mayor y a la otra menor.
    \item[ Principio (del lat\'in principium): ] Argumento considerado como origen, fundamentaci\'on o raz\'on primera de un razonamiento.
    \item[ Tesis (del lat\'in thesis, y \'este del griego): ] Conclusi\'on o proposici\'on que se mantiene con razonamientos.
    \item[Postulado (del lat\'in postulatus): ] Proposici\'on cuya verdad se admite sin pruebas y que es necesaria para servir de base en ulteriores razonamientos.
    \item[Propiedad (del lat\'in propietas): ] Atributo o cualidad de una persona o cosa.
    \item[Teorema (del lat\'in theorema, y \'este del griego): ] Proposici\'on que afirma una verdad susceptible de demostraci\'on.    
\end{description}
\chapter{Combinatorics}
\import{Mathematics/Combinatorics/}{Combinatorics.tex}
\chapter{Probabilidad}
\import{Mathematics/Probabilidad/}{index.tex}

